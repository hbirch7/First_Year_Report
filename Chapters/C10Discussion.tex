In this first year report, a successful review of the cosmological indications for Dark Matter (\S\ref{Chap2:Cosmo}) and the possible candidates (\S\ref{Chap3:Candi}) preceded a description of the possible methods in which Dark Matter could be found (\S\ref{Chap4:Search}). A sufficient description of the LZ experiment (\S\ref{Chap5:LZ}) was followed by an explanation of how the liquid Xenon time projection chamber will detect a WIMP (\S\ref{sec:TPC}). The veto system that will be used to reject and characterize the background radiation from the surrounding environment was then outlined. 
\newline
The Outer Detector Optical Calibration System developed and design by the High Energy Particle group at the University of Liverpool was illustrated, as well as the testing procedures that were conducted during the development stages (\S\ref{Chap6:OCS},\ref{Chap7:FST}). The components designed by the group were also described (\S\ref{Chap8:FSS}) leading to an investigation into the development of a insert for the FSS to prevent creep in the fibre as a force was applied (\S\ref{sec:FSSID}).  
\newline
The group are currently on target to deliver the OCS electronics system to the experiment. Before the system is delivered to the experiment, further testing of the control system will take place due to a final firmware update. The firmware update will include all of the discussed capabilities with the added function to record the temperature of each individual pulser board. The method in which the system will communicates to the OCCs has been changed. Throughout the testing of the OCCs the computer was only able to communicate with a single board. The final firmware update will assign one OCC as a "master" and the remaining OCCs as "slaves". The OCCs will be connected together and the control commands will then pass from the computer to the master OCC then onto the slave OCCs. The reverse will be used for data read back from the system.
The Optical Calibration program is still under development and has a planned completion date of September 2019. The program is being built using Ignition Designer. The user interface is written in Java and will use a SQL database to store calibration parameters and other key constants. The program will use python scripts to communicate with the OCCs and will have a selection of calibration scripts which will be used by operators of the detector during calibration runs. 

\section{Outlook}
For spin-dependent WIMP-neutron(-proton), a sensitivity of $2.7\times10^{-43}cm^{2}$ ($8.1\times10^{-42}cm^{2}$) for a $40\:GeV/c^2$ mass WIMP is expected to be achieved by the LUX-ZEPLIN Experiment \cite{WIMPsense}. The full characterisation of signals detected by the experiment is key to achieve such sensitivity. The addition of an Outer Detector and extensive veto system will give the experiment the ability to reject signals produced from background radiation. The Optical Calibration System produced by the group at Liverpool will ensure that the Outer Detector will meet it's required sensitivity allowing the LUX-ZEPLIN Experiment to push the WIMP-nuclear cross-section ever close to the neutrino floor. 