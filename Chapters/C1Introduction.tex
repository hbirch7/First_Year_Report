Over the last $\sim150$ years physicists have observed discrepancies between astronomical measurements and the observations of luminous matter (\S\ref{Chap2:Cosmo}). Many postulate that the existence of Dark Matter in our Universe is the reason for these discrepancies. Application of the virial theorem along with measurements of galaxy velocity have been used to determine the mass of galaxies, such calculations differ to expected mass given the observed luminosities. Similar differences can be seen when using gravitational lensing, where the mass of an astronomical objects be reconstructed and the expected mass determined by the object's luminosity differ. The ratio of mass-light can be used to estimate the amount of Dark Matter in cosmological structures throughout our Universe. The distribution of energy throughout our universe can be also estimated using cosmological models such as the $\Lambda$CDM model. Using the most recent observations \cite{Planck2018} and measurements \cite{WMAP}, the energy distribution of Universe can be estimated to be: $\sim5\%$ Baryonic matter;$\sim26\%$ Dark Matter; and $\sim69\%$ Dark Energy. With Dark Matter constituting over a quarter of the energy/mass in our Universe, these estimates provide the motivation to investigate and understand it's properties. 
\newline
Of the possible candidates for Dark Matter (\S\ref{Chap3:Candi}), Weakly Interacting Massive Particles (WIMP's) are currently seen to be one of the more viable candidates by the scientific community. The LUX-ZEPLIN (LZ) experiment will probe the parameter space of the WIMP via direct detection (\S\ref{sec:directdetection}) using a liquid noble gas approach. The experiment aims to push the limits of the WIMP cross section close to the neutrino floor. Building on the success of it's predecessor, LUX \cite{LUX}, LZ will utilise a liquid Xenon time projection chamber (LXe-TPC) (\S\ref{sec:TPC}) and hopes to achieve greater sensitivity with the addition of Outer Detector (OD). The OD will be instrumented as a veto system to rejected unwanted signals within the LXe-TPC. (\S\ref{sec:veto}). The High Energy Particle group at the University of Liverpool are responsible for the development and production of the Outer Detector Optical Calibration System (OCS).
\newline
This report will present a short literature review to outline the motivations to search for Dark Matter, the possible candidates and the methods in which the candidates could be found. Then the detection method LZ will utilise and it's veto system will be described (\S\ref{Chap5:LZ}). The Outer Detector Optical Calibration System will be described (\S\ref{Chap6:OCS}), followed by an explanation of the tests which were performed on the system throughout it's development (\S\ref{Chap7:FST}). The Fibre Support Structures (FFS) used within the OD to align the injection points will be demonstrated (\S\ref{Chap8:FSS}). An investigation will take place into the development of an insert that will be used in the FFSs to hold the fibre in place (\S\ref{sec:FSSID}). A discussion about the future development of the OCS will also be made (\S\ref{Chap10:Discussion}). 