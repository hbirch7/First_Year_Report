One probable solution to explain the discrepancies in the mass to light ratio discussed in chapter \ref{Chap2:Cosmo} is to assume there is no Dark Matter present, and a modification of Newton's gravitational laws is needed to yield the observed results. Modifications such as MOND \cite{MOND} and TeVeS \cite{TeVeS} attempt to solve the problem in a classical and relativistic manner respectively, both failing to do so independently. MOND fails to explain the observations on the structure of CMB whilst also violating some of the fundamental laws. Whereas TeVeS successfully solves the problems encountered by MOND but fails to describe rotation curves and gravitational lensing.
Another possible explanation for large mass to light ratios could be the presence of Massive Astrophysical Compact Halo Objects (MACHOs) which emit little to no radiation. MACHOs could come in the form of neutron stars, black holes, brown dwarfs or un-associated planets. No suitable MACHO candidates have been found using gravitational microlensing \cite{MicroLens}. Baryonic nature of Dark Matter can also be ruled out due the Big-Bang nucleosynthesis (BBN)\cite{BBN}.
\newline
One of the more widely accepted and well motivated candidates for Dark Matter is in the form of a Weakly Interacting Massive Particle (WIMP). An assumption is made such that in the early universe WIMP's were in equilibrium with the cosmic plasma. During the expansion of the universe, the plasma cooled to a temperature lower than that of the WIMP mass leading to the decoupling from the plasma \cite{DMProd}. The Dark Matter relic density was reached at this freeze-out temperature when the WIMP annihilation rate became less than the Hubble expansion rate. The cross-section needed to detect a particle in the current Dark Matter density is of the order of the weak interaction scale, giving theoretical reason for motivation to search for a WIMP-like particle. 
\newline
The only suitable candidate from Standard Model to be considered would be the neutrino. It would be be thought as a "hot" Dark Matter candidate due to it's relativistic velocity in the early universe. But due to neutrino's fermionic nature and constraints set by the Fermi-Boltzmann distribution, they cannot account for the Dark Matter density within halos \cite{DirectDetection2015}. The hypothetical, Sterile Neutrino, introduced to explain the size of a neutrino mass could provide a possible candidate for Dark Matter. They could either be a cold (always non-relativistic) or a warm (only relativistic in the early universe) Dark Matter candidate depending on their production mechanism. Unfortunately due to the low mass and interaction strength a sterile neutrino would be difficult to detect directly. 
\newline
As well as WIMP's and neutrinos, axions are another form of particle which could be a candidate to solve the Dark Matter problem. The axion particle was postulated in 1977 \cite{ConsvCP} to solve the "strong CP-problem" due to P and CP violation from the experimental bound on the neutron dipole moment \cite{NeutronDi}. Axion-like particles would have been produced in the early Universe in mechanisms such as vacuum realignment \cite{InvisAx}. Due to a small free streaming length, the axion is considered a "cold" Dark Matter candidate. 
\newline
The candidates discussed in this chapter were not originally proposed to solve the Dark Matter problem, but with the motivation to provide solutions to other models and theories in physics. This external motivation only strengthens the importance of the candidates. A more detailed discussion of of the evidence and candidates for Dark Matter can be found in \cite{DMCandidates}.