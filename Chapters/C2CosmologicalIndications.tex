Historically, the term "Dark Matter" was incorrectly used to explain discrepancies in measurements of the movement of stars within the Milky Way Galaxy by Jacobus Kapteyn. In 1933 Fritz Zwicky observed unexpectedly high velocities of nebulae in the Coma cluster. He applied the virial theorem to the observations to find that there was 400 times more mass calculated than was visibly observed. The effect of gravity in the cluster was also far too small for visible matter which was moving at such high velocities. After taking into account the discrepancy in mass and gravity, Zwicky inferred the existence of Dark Matter which would "hold" the cluster together \cite{Zwicky1933}. 
\newline
Another indication for existence of Dark Matter is the observed effect of gravitational lensing \cite{GravLens}. It was first described by Albert Einstein in 1936 \cite{Einstin1936} and was further studied by Zwicky in 1937 \cite{Zwicky1937}. The effect occurs when a massive object is situated in the line of sight between the observer and the object being observed. Due to the gravitational field of the massive object, light rays deflect from their path resulting in a deformation or in multiple images of the object being studied. The gravitational potential of the massive object can be reconstructed using the degree of deformation of the light. The mass of the object in the line of sight can also be reconstructed using this method and from observations it is found that the reconstructed mass is 200 times greater than the luminous matter observed.
\newline
Further discrepancies in the mass to light ratio measured by Horace W. Babcock in 1939 in his study of galaxy rotation curves \cite{RotationCurves}. His study of the Andromeda Galaxy suggested that the mass to light ratio increased radially due to light absorption within the galaxy. It was not until Jan Oort discovered a non-visible halo, NGC 3115, in 1940 which could be used to explain rotation curves within Andromeda \cite{Oort1940}. In 1978 Vera C. Rubin \textit{et al.}\cite{Rubin} studied a sample of 10 high-luminosity spiral galaxies, to find that the rotation velocities of stars in the galaxies would remain constant as the distance from the galactic centre increased. This was a contradiction to the expected result from Newtonian Dynamics which states that objects outside the visible mass distribution should have velocities $v \propto 1/\sqrt{r}$. The presence of a Dark Matter halo distributed uniformly throughout a galaxy could explain the observed phenomena.
\newline
The presence of Dark Matter in our universe can be shown using the $\Lambda$CDM model which consists of six parameters \cite{LCDMparam}. The model takes into account high sensitivity small angular resolution measurements in the infra-red spectrum from the Planck space observatory \cite{Planck2018} and fluctuations in Cosmic Microwave Background (CMB) radiation early on in our universe, observed by the Wilkinson Microwave Anisotropy Probe (WMAP) \cite{WMAP}. The model is fitted to the cosmological observations and measurements to determine some of the model's parameters.
The model's six parameters are physical baryon density parameter, physical Dark Matter density parameter, the age of the universe, scalar spectral index,  the amplitude of the initial fluctuations and re-ionisation optical depth \cite{LCDMparam}.The most recent estimates from Planck 2018 \cite{Planck2018} allow a calculation of the mass-energy distribution in our universe. The analysis shows a spatially-flat flat universe with $\Omega_B = 0.049$, $\Omega_{DM} = 0.265$ and $\Omega_\Lambda = 0.686$ representing the densities of baryonic matter, Dark Matter and Dark Energy. 
\newline
It is postulated that the anisotropies of the CMB arose from quantum fluctuations in matter during inflation of the early universe. The concentration of Dark Matter density increased as the fluctuations propagated through the universe during the inflation, as the baryonic matter in the concentration cooled cosmological structures formed due to increasing gravitational forces. The evolution of the Universe has been simulated and has accurately reproduced the measurements produced by galaxy surveys such as those of the Lyman-$\alpha$ Forest. Simulations have also reproduced the effects of weak lensing to confirm the cosmic structure taking into consideration non-luminous and non-baryonic matter as well as galaxies and gas clouds \cite{GravLens}. 